We show that whp,
\begin{align}
\fc{\diam\pat{Giant}}{\log_d n} &\le C
\end{align}
Assuming that  $Z_r$ is not too small (large constant times $\log n$) and the total number of vertices we've explored is not too large ($o(n)$), it's very unlikely that $Z_{r+1}<(d-\ep)Z_r$. We expect it to grow by a factor $d$. The probability is small enough for us to take a union bound
\begin{align}
\Pj\ba{Z_{r+1}<(d-\ep) Z_r|
B\log n < Z_r, \sumz ir Z_i \le n^{\fc 34}
}&\le n^{-3}.
\end{align}
Use the Chernoff bound on 
\begin{align}
Z_{r+1} &\sim
\Bin\Big(
n - \ub{\sumz ir Z_i}{\approx n^{\fc 34}} , \ub{1-\pa{1-\fc dn}^{Z_r}}{\approx dZ_r/n}
\Big)\\
\E[Z_{r+1}|Z_r] &\ge (d-o(1)) Z_r.
\end{align}•
Once we get some level that's big enough, then we'll keep growing by a factor of almost $d$ each time. We just need to make sure we can get a reasonably big level without waiting too long.

For the second step, think about the exploration process to control sizes of components.
Let $A_t$ be the number of active vertices (vertices we found but whose neighbors we haven't explored). The number of vertices behave like a biased random walk.
On average at each step find $d$ new vertices and use up one vertex, you find more than you've explored, so this is like a biased random walk.
We were just interested in how many vertices, not the way they're arranged. But we can explore in a specific order; that won't change any calculations.

Do a breadth first search: Start at root, explore all of its neighbors. Don't go to level $r+1$ until we've explored everything in level $r$. After we finish level $r$, the ones in level $r+1$ will all be active.
Now we can use bounds on $A_t$.
\begin{align}
A_t &\ge \wh S_t \sim \pat{biased random walk}\\
\Pj(0<A_t<\de t) &\le e^{-c_*t}.
\end{align}
Let $r_*$ be the first level with $Z_{r_*}\ge B\ln n$. Then every level after that will grow by at least $d-\ep$ whp. 

Claim: Whp, $r_*\le \pa{\fc{B}{\de}+\fc{2}{C_*}}\log n+1$.

Suppose not, that we got to level $r=\fc{B}{\ep}\log n\vee \fc{2}{C_*\log n}$. 
%Constant times $\log n$.
%$Z_r$  
Then $Z_{r+1} \ge \de \fc{B}{\de}\log n$

We control what happens in 2 phases:
We need burn-in time to have level with size $\fc{B}{\de}\log n$, and enough levels so that when we multiply by $d-\ep$ each time we get to something on the order of $n$.
%Control what happens in 2 phases: Don't wait too long to get to level with $\fc{B}{\de}\log n$, then keep growing with size $d-\ep$. 
We want each with high probability to take union bound over all vertices in the graph.
\begin{align}
Z_{r_*+\ell} &\ge (d-\ep)^\ell \log n
\end{align}
Let $\hat r = \min\set{r}{Z_r\ge n^{\rc 2+\ep}}$. With high probability \begin{align}\hat r \le \pa{\fc B{\de} + \fc{2}{c_*}}\log n + \pa{\rc 2 + \ep}\log_{d-\ep}n + 1\le C\log n.\end{align}

Either they intersect or not, in which case we calculate the number of edges between them.
The number of connections is $\Bin((n^{\rc 2 + \ep})^2, \fc dn)$, which has mean $n^{2\ep}\to \iy$. 
\begin{align}
\Pj(\Bin((n^{\rc 2+\ep})^2, \fc dn)=0) = \pa{1-\fc dn}^{n^{1+2\ep}}\le e^{-n^{2\ep}}.
\end{align}•

Where did we use the fact that the vertex is in the giant component
Some vertices are isolated, this can't be true for all vertices. The number of active vertices can die out somewhere at the beginning and the process stops. We assume this doesn't happen, $A_t>0$.
%Somewhere along the line 

The diameter is $\le \max_{u,v} \wh r_u + \wh r_v + 1\le 2C \log n+1$ whp.

What are the similarities and differences for distances in the Erd\H os-Renyi and the $d$-regular graph?

\subsection{Random $d$-regular}
Whp these are connected, assuming $d\ge 3$.

We don't expect to see many cycles, so the neighborhood should look like a $d$-regular tree, $Z_r=d(d-1)^{r-1}$.

The typical distance and diameter converge to 1; things are more concentrated.
\begin{align}
\fc{d(u,v)}{\log_{d-1}n} &\xra p1\\
\fc{\diam(G)}{\log_{d-1}n} &\to 1.
\end{align}
This wasn't true for the Erd\H os-Renyi graph because some vertices got to a bad start; this doesn't happen here.

%may be 1 cycle, else look like a $d$-regular tree
\begin{lem}
%connected whp
A ball of radius $\rc{5}\log_{d-1}n$ has at most 1 cycle for all $u\in G$ whp.
\end{lem}
%hard to split off and form own component.
%It branches nicely like a tree
%up to this radius
%at any point in the graph
%whp
Everyone who works in random graphs knows some theorem like this to be true. It gets reproven a lot because people don't know where to look up the fact.\footnote{I solved some problem, a couple of years later people have cited it. They were all just citing this lemma. You don't always get cited for the things you want.}%i know where to look.
%HL: moral: be useful. Write nice lemmas.

Let $I_j$ be the indicator that the $j$th half edge is matched to a new vertex. The number of vertices we've found so far is at most $j$, so the number of remaining half-edges is at least $d(n-j)$. 
\begin{align}
\Pj(I_j ) &\le \fc{dj}{d(n-j)} \le O\pf jn.
\end{align}
The number of times we form cycles is stochastically dominated by (using a coupling argument)
\begin{align}
\sumo j{d(d-1)^r} I_j &\le \Bin\pa{d(d-1)^r, O\pf{d(d-1)^r}{n}}\\
&\le \Bin(n^{\rc 5}, n^{-\fc 45})\\
\Pj(\sum I_j \ge 2) &\le O\pa{\binom{n^{\rc 5}}2 n^{-\fc 45}}\le O(n^{-\fc 65}).
\end{align}
%\E is $n^{-\fc 35}$. Main contribution from =2
%Bound by number of Bernoulli trials.
We got an exponent $<-1$ so we can take a union bound over all vertices in the graph.
You only get a cycle in the first $\rc 5\diam$. This is a worst-case calculation. We could have done the same calculation for $r=\rc 2 \log_{d-1}n$. Then we get
\begin{align}
\sum I_j &\le \Bin(n^{\rc 2}, n^{-\rc 2})
\sim \Pois(O(1)).
\end{align}
%radius that gives you $\sqrt n$ vertices.
Continuing %?
$
\le \Bin\pa{d(d-1)^r, \fc{d(d-1)^r}n}
$
which has mean $\fc{(d-1)^{2r}}{n}\ll (d-1)^r$ if $r<(1-\ep)\log_{d-1}n$. Whp $Z_{r+1}\ge (d-o(1)) Z_r$. 

For a typical vertex,
\begin{align}
Z_r &\ge (1-o(1)) d(d-1)^{r-1}\\
Z_n &\le d(d-1)^{r-1}.
\end{align}
%about max size if could be
%look at growth
For all $v\in G$,
\begin{align}
Z_r &\ge C d(d-1)^r\\
\text{for }r&\le (1-\ep)\log_{d-1} n
\end{align}
(There could be a cycle at the beginning, so the constant in front of the exponential is smaller.)
%\fc 56 \cdot 3 \cdot 2^r, 3\cdot 2^r
Grow to $r_u=\rc 2 \log_{d-1} n$ and $r_v = \rc 2\log_{d-1}n+\ell$; the sizes are $(d-o(1))(d-1)^{r_u-1}$ and $(d-o(1))(d-1)^{r_v-1}$.
Then
\begin{align}
\fc{(1-\ep)d(d-1)^{r_u}}{dn}
\Pj(I_j) &\le \fc{(d-o(1))(d-1)^{r_v}}{(d-o(1))n}\\
\sum_j I_j &\le \Bin\pa{d(d-1)^{r_u} , \fc{(d+\ep)(d-1)^{r_v}}{dn}}\\
\sum_j I_j &\ge \Bin \pa{d(d-1)^{r_u} , \fc{(d-\ep)(d-1)^{r_v}}{dn}}\\
\pat{\# connections} &\sim \sum I_j\\
&\sim \Pois\pa{\fc{d(d-1)^{r_u + r_v}}n}\\
&\sim \Pois \pa{\fc{dn(d-1)^\ell}n}.
\end{align}
%mean grow exp so conc around mean
This says that
\begin{enumerate}
\item
$\Pj\pat{no connections}\le e^{-c(d-1)^\ell}$
%sum of $\ell$ Poissons with mean 1.
\item
The number of connections is concentrated.
\end{enumerate}
\begin{align}
\Pj(d(u,v)>\log_{d-1}n +\ell)\\
&\le e^{-c(d-1)^\ell}.
\end{align}•
What's the most number of vertices you can have in a ball of radius $r$? When it's a tree, 
$\sumz ir d(d-1)^i \sim C(d-1)^r$.
The typical distance is $\log_{d-1}n$ and the fluctuations will typically be a constant.

Note that $\ell$ can be doubly exponentially small here, so we only need to take $\ell=\log\log n$.

This shows 
\begin{align}
\diam(G)\le \log_{d-1} n + C\log\log n.
\end{align}
(If you take $n$ to be the number of particles in the universe, $\log\log n\approx 5$.)
%geometrically, looking at graph, size of boundary only decreases by constant factor.

%You won't get exactly max but only off, e.g. by factor 5/6. Can take $C=\rc2$ or $\rc 4$.
%up to radius 1/5\log n, boundary big, so only see negligible number of new cycles each time grow depth by 1. whp all vertices.

We will see that things become very concentrated for $d$-regular graphs. The difference between max and typical distance is only $\log\log n$.

We can also do all of this for random graphs with other degree distributions. If it's not regular, it looks more like the Erd\H os-Renyi graph, particularly if there are vertices of degree 0 or 1. Suppose $d_v\sim \nu$ are iid and 
\begin{align}
\Pj(d_v = k) &\sim C k^{-\al-1},
\end{align}
they decay like some power. Then $\E d_v<\iy$ if $\al>1$ and $\Var(d_v)<\iy$ if $\al>2$. (If $\al<1$, the max degree greater than $n$, it doesn't make sense.)
When $\al>2$, the whole story looks similar to the Erd\H os-Renyi case. The interesting variation is what happens when $1<\al<2$, finite mean and infinite variance.
The neighborhood looks like a modified GW branching process. The root is distributied according to $\nu$ and the children have distribution $\sim \nu^*$, 
\begin{align}
\Pj(d_{\nu^*} = k) &=\fc{k\cdot \Pj(d_\nu = k)}{\E d_\nu} \sim C_1 k^{-\al}.
\end{align}•
The size-biased distribution has infinite mean.
\begin{align}
Z_{r+1} &= \sumo i{Z_r} d_i^*,& d_i^*&\sim \nu^*\\
\sim \max_{1\le i\le Z_r} d_i^* &\sim Z_r^{\rc{\al-1}}\sim Z_r^{\rc{\al-1}}\\
Z_r &\sim e^{c\prc{\al-1}^r}.
\end{align}
Rather than each level growing like a constant factor, it grows like this power. %doubly exponentially small
We only need $r\sim \log \log n$ because $Z_r$ is doubly exponentially in $r$. Random graphs with heavy-tailed distributions have much smaller distances; neighborhoods grow really quickly. %lots of details. this is moral.