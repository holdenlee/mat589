\section{Distances in $G(n,\fc dn)$}

If you take 2 random vertices, how far apart are they in terms of graph distance? I.e., what is the shortest path between them?

Locally $G(n,\fc dn)$ looks like a branching process with $\Pois(d)$ offsprint, growth rate $d$. Letting $Z_r$ be the number of vertices up to depth $r$, $Z_r\sim d^r$.

Just looking at the exponential growth rate, in order for $d^r \ge \ep n$, we need $r\ge (1-\ep)\log_d n$. 

\begin{align}
\E[Z_r|Z_{r-1}] &\le \E[\Bin(Z_{r-1}, \fc dn)|Z_{r-1}]\le d Z_{r-1}\\
\implies
\E Z_r &\le d^r.
\end{align}
%Upper bound because 2 vertices can connect to the same vertex.
If $r=(1-\ep)\log_d n$ then $\E\sumz ir Z_i\le \sumo i{(1-\ep)\log_dn}d^i = O(n^{1-\ep})$.
This shows an upper bound; it turns out this is actually the right bound.
\begin{thm}
If $d>1$, conditioned on $u,v$ being in the giant component,
\begin{enumerate}
\item
$\fc{d(u,v)}{\log_d n}\xra p 1$.
\item
$\fc{\diam\pat{Giant}}{\log_d n}\in (1+\de, C)$ w.h.p. 
\end{enumerate}
\end{thm}
Actually, with a more refined analysis we can show the latter converges to some constant $C_d$.

We have to be more careful about the connection between the graph and the branching process---they are only close up to some radius. At large enough radius, it stops looking like a tree.

\begin{lem}
If $Z_n$ is a BP with offspring distribution $X$ and $\E X=\mu$, $\Var(X)=\si^2<\iy$, then $\mu^{-n} Z_n\to W$ a.s. and $\Pj(W>0)=\Pj\pat{BP survives}$. 
\end{lem}
\begin{proof}
Recursively calculate
\begin{align}
\E[Z_r|Z_{r-1}]&=\mu Z_{r-1}\\
\Var[Z_r|Z_{r-1}]%&=\Var[\sumo in X_i|Z_{r-1}]\\
&=\si^2 Z_{r-1}\\
\Var(\mu^{-r} Z_r - \mu^{-r+1}Z_{r-1})
&= \E [\Var(\mu^{-r}Z_r - \mu^{-(r+1)}Z_{r-1}|Z_{r-1})]
+ \cancel{\Var(\E(\cdots | Z_{r-1}))}\\
&= \E[\si^2 \mu^{-2r} Z_{r-1}] = \si^2 \mu^{-(r+1)}.\\
\sumo r{\iy} \ve{\mu^{-r} Z_r - \mu^{-(r-1)}Z_{r-1}}_2 &<\iy\\
%martingale
\implies \mu^{-r}Z_r &\xra{\text{a.s.}}w\\
1 = \E \mu^{-r}Z_r &\to \E W=1\\
\Pj(W>0) &=a>0.
\end{align}
The reason for this calculation is to get the last part.
If the branching process didn't survive, the limit is 0. We need to show that when it does survive, we get some nonzero process.

%every vertex independent, each individually probability $\al$ of having dense enough subtree, where limited density is positive.
\begin{align}
\Pj(W>0) &>\E(1-\al)^{Z_r}\to \Pj(Z_r\not\to \iy)
\end{align}
Let $R=\min\set{r}{Z_r\ge M}$. Then $\Pj(W=0|R_m<\iy) \le (1-\al)^m$. 
When the process survives, there is always such a level. Each time it's of size $\le M$, there's some fixed probability it died out. The probability that it always stays below and never dies out is 0. (Cf. A Markov chain with an absorbing state, if it never reaches the absorbing state it must be transient.)

If the branching process survives, then $R_m<\iy$ w.p. 1. Then $\Pj(W=0)\le (1-\al)^m$.
%we can choose any $m$ here

So $\Pj(W=0,\,W\text{ survives})=0$.
\end{proof}

%\Var(X) = \E(\Var(X|Y)) + \Var(E(X|Y))

The idea of coupling is as follows. Let $\wh X, \wh Y$ have the same probability distributions as $X$, $Y$. %be on the same probability spaces. 
We want to maximize $\Pj(\wh X=\wh Y)$ in total variation distance
\begin{align}
d_{TV}(X,Y) &=\min_A \Pj(X\in A)-\Pj(Y\in A)\\
&=\rc 2 \sum_n |\Pj(X=n) - \Pj(Y=n)|\\
&= \min_{\text{coupling }\wh X,\wh Y} \Pj(\wh X \ne \wh Y).
\end{align}•

Another fact: if we have 2 sequences of random variables $(X_i)$, $(Y_i)$ and they are independent, then 
\begin{align}
d_{TV}(\sum X_i, \sum Y_i) &\le \sum_i d_{TV}(X_i,Y_i).
\end{align}
Construct the optimal coupling individually. The TV distances are the probabilities an individual one fails; use a union bound.

I want to make more precise how the Erd\H os-Renyi graph is like a branching process, in terms of the total variation distance.
%local neighborhood of a vertex

We couple the first level, Binomial and Poisson. Then we can reveal the next level.
%If each time we can successfully couple, then 
Continuing, we a coupling of the first $r$ levels of the GW random process and the local neighborhood in the ER random graph. I want a quantifcation of when this starts to fail.

For $\Pois(p)$ and $\text{Ber}(p)$,
\begin{align}
d_{TV} (\text{Ber}(p), \Pois(p))
&= \rc 2 (|e^{-p}-(1-p)| + |pe^{-p}-p| + \Pj(\text{Pois}(p)\ge 2))=O(p^2)\\
\implies 
d_{TV}(\Bin(n,p), \Pois(np)) &\le O(np^2)
\end{align}
by thinking of $\Bin(n,p)$ as a sum of $n$ iid $\text{Ber}(p)$'s and $\Pois(np)$ as a sum of $n$ iid $\Pois(p)$'s. But we actually have $\Bin(n-k,p)$, so we consider
%with $k$ extra trials
\begin{align}
d_{TV}(\Bin(n,p), \Bin(n-k,p))&\le kp.
\end{align}
(Bound by the probability that the last $k$ trials aren't all 0.)
Putting together, \fixme{??}
\begin{align}
\Pj\pat{coupling fails with $k$ levels}
&\le \sum_j O\pa{a_j \cdot \fc dn} + O\pa{n\cdot \fc{d^2}{n^2}}
\le O\pa{\pa{\sumo ik Z_i}^2}.
%O\pa{\rc n d^k} + O(d^k\cdot d\cdot \rc n) \le O(d^k/n) = o(1)$ if $k\le (1-\ep \log_dn$
\end{align}•
The main contribution comes from the last one.
%number of vertices seen so far.
%d^k
This will work up to the point when $d^k$ is about $\sqrt n$.

We can keep the coupling going with good probability until we see about $\sqrt n$ vertices.
%swept under the carpet. 

Given $f_r = \si(Z_1,\ldots, Z_r)$, What is $Z_{r+1}$ given $f_r$? 
%2 edges connect to the same vertex.
\begin{align}
\Bin(Z_r (n-\sumo i r Z_i ),\fc dn) - ``\text{cycles}''
\end{align}
%$\E Y \approx dZ_r$, $Y-\E Y$ small.
Once $Z_r$ is big, this is concentrated. 
By the Chernoff bound,
\begin{align}
\Pj(|\Bin(Z_r,(n-\sum z_i), \fc dn)- \E(\cdot)|\ge x|f_r)
&\le \exp\pf{-x^2}{2Z_r n \fc dn (1-\fc dn)}\\
&= o\prc n, &x &\ge \sqrt{3d Z_r \ln n},
\end{align}•
%big constant times log n, take $x$ that is small compared to $Z_r$, and this fluctuation from the binomial won't matter much
good if $Z_r \ge B\ln n$. 
Run coupling up to $r=\pa{\rc 2 - \ep}\log n$ for both nodes. There are $(\rc 2 + \ep) \lg_d n$ nodes. Say there are $W_u n^{\rc 2+ \ep}$ and $W_v n^{\rc 2+\ep}$ nodes with $W_u,W_v>\de$. 
%sum on order of $n$
Then
\begin{align}
\Pj[\#\text{ connections}=0] &\le \Pj[\Bin(n^{1+\ep},\fc dn)=0]\le 
(1-\fc{d}{n})^{\de^{2} n^{1-2\ep}}
e^{-\de^2 n^{2\ep}}\to 0.
\end{align}
%Z_r \ge n^{\rc 2 - \ep}

%To control the number of cycles: 
After level $(\rc2 -\ep)\log_2 n$, we use another argument based on the fact that binomial random variables are concentrated around mean.

We want to get to size $(\rc 2+\ep)\log_2n$ because of the birthday paradox. If we have $\sqrt n$ on either side, then about $O(1)$ vertices will be connected to each other, if we have more on either side, then whp some pair of vertices will be connected to each other.

How big should the radius be before you start seeing cycles? At $\rc 2 \log_2 n$. 

The number of cycles is about $\Bin\pa{\binom{Z_k}2, \pf dn^2}\approx \rc 2 \fc{Z_r^2} n\ll \fc{Z_r}n$.
%\ll Z_k$.

\begin{align}
\Pj[d(u,v)\ge (1+\ep)\log_d|u,v\in\text{Giant}]&\to 0.
\end{align}•

%log n. some vertices 1+\de log n.
%where did we use a typical vertex
If we make $\de$ small enough, the constants in front of the branching process will both be $>\de$. This will be true for 2 randomly chosen ones. What if you take $n$, what is the smallest nonzero value you get? The minimum tends to 0. 
There are some vertices for which the branching process survives but gets a very slow start, ex. it has one child for $\de \log_d n$ steps.
This happens for a node with probability $(de^{-d})^{\de \log_d n} \approx n^{-\de(d-\log d)/\log d}$. %typical dstance $(1+\de)\log_n$
%unlikely for a node
%but happen fairly frequent
It's like a branching program delayed
$\de \log_d n$ steps.

It comes down to the structure of the giant component. Take the part that's like an expander. Hanging off it are a bunch of trees, looking like subcritical branching proceeses (but you'll still get order $\log n$).%It's a well-connected graph plus t
%subcritical bp's

%scale-free graphs, loglog n