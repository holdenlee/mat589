\chapter{Stochastic processes on random graphs}

\section{Random walk on random graphs}

Consider the random walk given by $P_{vu} = \rc{d_v} I(u,v)$ where $I(u,v)$ is the indicator variable for $uv$ being an edge.

The stationary distribution is $\pi_v \sim \fc{d_v}{2|E|}$ We check that this satisfies detailed balance:
\begin{align}
\pi_vP_{v,u} &= \pi_u P_{u,v}
=I(u,v).
\end{align}
On every edge of the graph, at stationarity, it's equally likely to go over the edge in each direction.

Let $X_t$ be the probability distribution after $t$ steps. $X_t^v$ is the distribution when started at $t$.

Question: how quickly does $X_t\to \pi$?
For the Erd\H os-Renyi graph, we just consider the giant component.

The standard quantity is mixing time in total variation distance.
\begin{align}
d_{TV}(X,Y) &= \rc 2 \sum |\Pj(X=k) - \Pj(Y=k)|.\\
t_{\text{mix}}(\ep)&= \min\set{L}{\max_v d_{TV}(X_t^v, \pi)<\ep}.
\end{align}
%\text{mix}
We'll also be interested in how $d_{TV}$ looks on the way down. Consider a sequence of growing random graphs.
There are two possible cases:
\begin{enumerate}
\item
Cutoff: $\fc{t_{\text{mix}}^{(n)}(\ep)}{t_{\text{mix}}^{(n)}(1-\ep)}$.

This is what the picture looks like for the random $d$-regular.
\item
Gradual decrease: This is what the picture looks like for Erd\H os-Renyi random graph. It looks like this because there are a few vertices that are slow starting points for the Markov chain; if it started at a more typical point, there would be a cutoff.
\end{enumerate}

I'll state some of the main definitions and tools from the theory of Markov chains.

Because the Markov chain is reversible, we can write it as follows. Let $D_\pi = \diag(\pi)$. $D_\pi^{\rc 2} PD_\pi^{-\rc 2}$ is symmetric so has all real eigenvalues. The biggest one is $\la_1$. If the graph is connected, the second one is strictly $<1$. There will only be an eigenvalue $\la_n=-1$ if the graph is bipartite. The graphs we're looking at won't be (whp). 
\begin{align}
1=\la_1>\la_2\ge \cdots \ge \la_n\ge -1.
\end{align}•
The spectral gap is $\ga = 1-(\la_2 \vee|\la_n|)$ (take the one with higher absolute value).
(A way to get around this is to look at the lazy random walk, where with probability $\rc2$ the walk does nothing. The transition matrix is $\rc2 (I+P)$. This also gets around the graphs being bipartite. The chain is aperiodic if the graph is bipartite.)

Running for $t$ steps corresponds to raising the transition matrix to the $t$th power.

The ratio goes to 1 by the contribution of the first eigenvector; the rest decay exponentially,
\begin{align}
\fc{P_t(x,y)}{\pi(y)}&=\sumo jn f_j(x) f_j(y)\la_j^t = 1+\sum_{j=2}^n f_j(x)f_j(y) \la_j^t.
\end{align}•
In 1980, people said, there's nothing more to think about in terms of finite Markov chains.
But what doesn't this take you is how long it takes for the distributions to be close together---the cutoff. It's only an asymptotic result. Persi Diaconis looked at this in terms of random walks on groups. The most famous example is shuffling a deck of cards. That brings a lot of representation theory of the symmetric group.

But here we'll think about the structure of these random graphs.

By Jensen's inequality,
\begin{align}
d_{L^2} (X_t^x, \pi) &=\sqrt{
\sum_y \ab{\fc{P_t(x,y)}{\pi(y)}-1}^2 \pi(y)
}\\
&\ge \sum_y \ab{\fc{P_t(x,y)}{\pi(y)}-1}^2\pi(y) = 2d_{TV}(X_t^x,\pi).
\end{align}
Bounding the $L^2$ distance controls the $L^\iy$ distance. 
\begin{thm}
\begin{align}
\pa{\rc{\ga}-1} \log \prc{2\ep}
&\le 
t_{\text{mix}}(\ep)
\le \rc{\ga}\ln \prc{\ep\pi_{\min}}.
\end{align}
where $\pi_{\min}$ is the minimum probability of a state in the Markov chain.
\end{thm}
For a regular graph, $\pi_{\min}=\rc N$, and you get the mixing time up to a factor or $\rc{N}$. Both bounds are tight.
\begin{df}
The \vocab{conductance} is
\begin{align}
Q(A,B) &= \sumr{x\in A}{y\in B} \pi(x) P(x,y)\\
&=\Pj(X_0^\pi\in A, X_1^\pi\in B).
\end{align}
The \vocab{conductance/Cheeger constant/bottleneck ratio} is
\begin{align}
\Phi:&= \min_{\pi(S)\le \rc 2}\fc{\min Q(S,S^c)}{\pi(S)}.
\end{align}
(This is $\Pj(X_1^\pi \in S^c|X_0^\pi\in S)$. Conditioned on starting in $S$, what's the probability you leave it in the next step?)
\end{df}
This is related to the surface-area-to-volume ratio.

\begin{thm}[Jerrum-Sinclair]
\begin{align}
\fc{\Phi^2}2 \le \ga\le 2\Phi.
\end{align}•
\end{thm}
In different cases, either the upper bound or the lower bound could be correct, up to constant factors.

How do we relate this back to graphs? 
\begin{df}
A graph $G$ is a \vocab{$\de$-expander} (edge expander) if
\begin{align}
\min_{|S|\le \fc{|V|}{2}}
\fc{|\pl S|}{|S|} 
\end{align}
where $|\pl S|=\#\set{(u,v)}{u\in S, v\in S^c}$.
\end{df}
For a regular graph, if it's an expander, the conductance is bounded below:
\begin{align}
Q(S,S^c) &= \sum_{a\in S, b\in S^c} \pi(a)P(a,b)
\\
&=\rc{d_n}|\pl S| \ge \rc{d^n} \de|S| = \fc{\de}d\pi(S)\\
\implies \Phi &\ge \fc{\de}d\\
\implies t_{\text{mix}} &\le
\fc{d}{\de} \pa{\log n + \log \prc{\ep}}.
\end{align}
This is the best you can hope for because the most number of vertices you can reach in $t$ steps is $d^t$.

\begin{thm}
If $h$ is random $d$-regular ($d\ge 3$) then there exists $\de_d>0$ such that $G$ is $\de_d$-expander whp  and so $t_{\text{mix}} = O(\log n)$.
\end{thm}
We've bounded $\Pj(\#\text{edges }S,S^c = \ell)$ through the configuration model. This gives
\begin{align}
\Pj\pat{$G$ is not $\de$-expander}
&\le \sumo k{\fc n2}\binom nk
\sumz{\ell}{\de k} \Pj_{\text{config}}(E(S,S^c)=\ell)\\
&\le \sumo k{\fc n2} \binom nk
\sumz{\ell}{\de k} \binom{dk}{\ell}
\binom{d(n-k)}{\ell}
\ell%'
M_{dk-\ell} M_{d(n-k)-\ell}\rc{M_{dn}}\\
&=\cdots \to 0
\end{align}
using Stirling's formula lots of times.

It turns out that $\la_2\sim 2\sqrt{d-1}/d$; if $d$ is large the spectral gap is large. It was an open question for  a long time. In about 2000 there was a 100-page proof of this asymptotic, $\la_2=\fc{2\sqrt{d-1}}d + o(1)$. 
A few years ago, there was a proof that was short enough to actually read.

These graphs are so nice that we can say more about them.

What should the constant in front of the $\log n$ be?

%Consider a random $d$-regular graph.
Let $\rh$ be the root, $X_0=\rh$. 
The speed of the random walk (in terms of the expected distance from the root) is $-\rc d + \fc{d-1}d$. Letting $D_t=d(\rh,X_t)$, $\fc{D_t}{t} \xra{\text{a.s.}} \fc{d-2}d$. (At the root the distance will only increase, but it will only go to the root a constant number of times.)

If this were a good heuristic, how long would it take to be spread uniformly? We need $d(d-1)^{\ell-1}\approx n$, $\ell\approx \log_{d-1}n$, which is just the diameter of the graph. How long does it take until the random walk reaches this? Multiply by $\rc{\text{speed}}$, to guess $\fc{d}{d-2} \log_{d-1}n$. The random walk on the $d$-regular graph has cutoff around this time.

We'll prove this two ways.

First, look at the non-backtracking random walk. At each step, choose a uniform neighbor to go to, except it can't go to the one it just came from. ($Y_{t+1}\ne Y_{t-1}$.) The non-backtracking random walk on the $d$-regular tree keeps going down the tree.

We can average over nonbackgracking random walks to say something about the original, $Y_{D_t}\stackrel d= X_t$. This is also true for any $d$-regular graph. 

%``Go back to the previous step and choose neighbor \#2.''
There are $d(d-1)^\ell$ non-backtracking walks of length $\ell$ in $T_d$. The same is true in $G$. We can have a 1-to-1 mapping of the two. We can map the infinite tree to $G$, $T_d\to G$ (CoverTree). You can map a walk on the tree to a walk on the graph, $\ph({}^{T_d}X_t) = {}^GX_t$. Thus it suffices to look at NB paths on the tree.

\begin{align}
\Pj(Y_t^v = u) &= \fc{\pat{\# NB path $v$ to $u$ of length $\ell$}}{d(d-1)^\ell}.
\end{align}
We worked this out when calculating the typical distance between 2 vertices.
\begin{align}
\pat{\# $\ell$-paths $u$ to $v$}
&\sim \Pois\pf{d(d-1)^\ell}{n}.
\end{align}
If $\ell-\log_{d-1} n\to 0$, then the Poisson is concentrated,
%We get a Poisson random variable whose mean is large and hence is concentrated.
\begin{align}
\fc{\#\ell\text{-path $u$ to $v$}}{d(d-1)^\ell}\xra p \rc n.
\end{align}•
The TV distance is
\begin{align}
d_{TV}(Y_t^v,\pi) 
&= \rc 2 \sum_d|\Pj(Y_t^v=u) - \pi_u|\\
&=  \sum_d\pa{\Pj(Y_t^v=u) - \rc{n}}^+\to 0.
\end{align}
where $x^+=x\wedge 0$ (the negative and positive parts are equal because they sum to 0).
You have to be careful starting from a vertex close to a short cycle. Run for a short period of time, then it won't be close to a cycle. You can run this argument and get the same result.

But we're interested in the actual random walk. 
\begin{align}
d_{TV} (X_t^v, \pi) 
&= d_{TV}(\sum_x \Pj(D_t=\ell) Y_t^v,\pi)\\
&\le \sum_\ell \Pj(D_t=\ell) d_{TV}(Y_t^v,\pi)\\
&\le \Pj(D_\ell \le (1-\ep) \log_{d-1}(n)) + o(1)\to 0.
\end{align}
We only have to worry about when the backtracking walk has taken to few walk for the NB walk to mix. But we know $\fc{D_t}{t}\xra{\text{a.s.}} \fc{d-2}d$. 

We need to say that before $\fc{d}{d-2}\log_{d-1}n$ the TV distance is big. If $t<(1-\ep)\fc{d}{d-2} \log_{d-1}n$, whp $D_t\le (1-\eph)\log_{d-1}r$, and so $X_t\in B_{(1-\eph) \log_{d-1}n}$ which only has $n^{1-\eph}$ vertices.

Recall that the local neighborhood looks like a tree until around $\rc \log_{d-1}n$. The further we go down, the more cycles. We can even compute how many we expect to see. Let's say we get down to $(1-\ep)\log_{d-1}n$. We see more but not so many---for a typical path, I don't expect any of the vertices to be in a cycle.


Let $X_t\sim \mu_t$.
Suppose we ran the Markov chain to $t\approx (1-\ep)\fc{d}{d-2}\log_{d-1}(n)$. 
$\mu_t = \wh \mu_t (1-\de) + \de v_t$, $\de=o(1)$, $\wh \mu_t(u)\le n^{-2\ep}$. 
We have exponential decay in $L^2$ distance.
%close to uniform on set $S$, but small measure, so still far from uniform measure. 
\begin{align}
d_{L^2}(\wh \mu_t, \pi)&\le n^{-2\ep}\\
d_{L^2}(\wh \mu_{t}P_r, \pi) &\le \ga^rd_{L^2}(\wh \mu_t, \pi)\to 0\\
\text{if }r &> C\ep \log_{d-1}n\\
d_{TV}(X_{t+r}, \pi) &\le (1-\de) d_{TV}(\wh \mu_tP_r, \pi) + \de d_{TV}(v_tP_r, \pi)=o(1)
\end{align}•
Only need to run for a small amount of time to 
make this go to 0.
$\pa{\fc{d}{d-2} - \fc{\ep d}{d-2} + c\ep} \log_{d-1}n. $
This concludes the second proof, which is also useful for Erd\H os-Renyi random graphs.
%cover for ER graph from typical vertex.
%random d-regular easier.