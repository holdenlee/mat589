\chapter{Random graphs}

\section{Models of random graphs}
\subsection{Erd\H os-Renyi random graph}
The simplest model is the Erd\H os-Renyi random graph $G(n,p)$. $n$ is the number of vertices, we are often interested in taking $n\to \iy$.

Most questions about the Erd\H os-Renyi graph are known. When is it $k$-colorable is a rare question that isn't known.

\begin{df}
The Erd\H os-Renyi random graph $G(n,p)$ is a graph with $n$ vertices, where every two vertices are connected with an edge with probability $p$ independently.
\end{df}
The degree of a vertex has distribution $\text{Bin}(n-1,p)$ which has mean $\approx np$.

To have a graph with more local structure, we allow $p$ to depend on $n$ rather than be constant. The sparse case is $G(n,\fc dn)$.

\begin{df}
The random $d$-regular graph is uniformly chosen over all $d$-regular graphs on $n$-labeled vertices.
\end{df}
More generally, a random graph with a given distribution $d_v$ is as follows: %draw $d_v$, the degree of vertex $v$, from a given degree distribution, then 
choose a graph uniformly from all graphs with that degree distribution. 

Two vertices are connected if there's a path of edges going from one to the other. 
A graph splits into connected components. 
We can consider local structure: the graph of vertices close to a given node.
On the other hand, diameter is a global property.

The degree distribution of $G(n,\fc dn)$ is $\text{Bin}(n-1,\fc dn)$, which is well-approximated by a Poisson distribution.
\begin{align}
\Pj(d_v=k) &= \binom nk \pf{d}{n}^k \pa{1-\fc dn}^{d-k-1}\\
&= \fc{(n-1)\fp{k}}{k!} \fc{d^k}{n^k} \pa{1-\fc dn}^n \pa{1-\fc dn}^{-k-1}\\
&\xra{n\to \iy} \fc{d^k}{e^{-d}}{k!} = \Pj[\text{Pois}(d)=k]
\end{align}
where $x\fp{y}:= x(x-1)\cdots (x-y+1)$.
That's the distribution for one node; what about the empirical distribution for the whole graph?
%\text{Pois}(d)
%
\begin{align}
N_k &= \rc{n}\#\set{v\in V}{d_v=k}\\  
\E N_k &= \rc n \sum_v \E \one_{d_v=k}.
\end{align}
I want this to be concentrated around the mean.

Are the degrees of the vertices independent? No. Two vertices share an edge.

I spend a lot of time taking things that aren't independent and showing that they behave as if they were independent.

Let $|A|=n^{\rc 3}$. 
Let $B$ be the event $\set{\exists u,v\in A}{(u,v)\in E}$, $A$ has an edge.
\begin{align}
\Pj(B) &\le \binom{n^{\rc 3}}2\fc dn \le \fc{dn^{\fc 23}}{n}\to 0
\end{align}
Most of the time there won't be vertices in $A$.
We needed $|A|\ll n^{\rc 2}$ to make $B$ happen with low probability.
 %this graph.
Conditioned on $B^c$, the degrees $\{d_v\}_{v\in A}$ will be independent of each other, and $\text{Bin}(n-n^{\rc 3},\fc dn)\approx \text{Pois}(d)$. Then 
\begin{align}
\Pj\ba{\rc{n^{\rc 3}}\#\set{v\in A}{d_v=k} - \fc{d^ke^{-d}}{k!}> \fc{\de}2}&\to 0.
%\le Ce^{-\de^2 n^{\rc 3}/2}
\end{align}
But I want the distribution on the entire graph, not just $A$, to be close to Poisson.

One way is to split into groups of size $n^{\rc 3}$, and apply the above. For this we want $\Pj \to 0$ with some rate of convergence.

Here's a simpler way.
The sample mean is close to the population mean.
For $n$ variables $d_v$, the mean of sampling (without replacement) $n^{\rc 3}$ goes to 0, 
\begin{align}
\Pj\ba{\ab{\rc n \sum_{v\in V} \one_{d_v=k} - \rc{n^{\rc 3}} \sum_{v\in V} \one_{d_v=k}}>\de}&\to 0.
\end{align}
To make this quantitative, use Azuma-Hoeffding to get this is $\le 2e^{-\de^2 n^{\rc 3}/2}$. We get $N_k\xra{P} \fc{d^ke^{-d}}{k!}$. 

What about the maximum degree? What is the maximum of $n$ iid $\text{Pois}(d)$ variables. Because of the $k!$, the tails of Poisson decay faster than any exponential distribution. The maximum looks like $\fc{\ln n}{\ln \ln n}$. To check this, plug this in and use Stirling's approximation.

For $k=\fc{\al\ln n}{\ln \ln n}$, 
\begin{align}
\Pj(\text{Pois}(d)=k) &=n^{-\al+o(1)}.
\end{align}
We actually want the same formula for the binomial distribution, but it holds by the same calculation.
By a union bound,
\begin{align}
\Pj(\max_{v\in V} d_v > (1+\ep) \fc{\ln n}{\ln \ln n})&\le nn^{-(1+\ep)}\to 0.
\end{align}•
%We know how to bound for a single vertex; we think they are more or less independent, and behave like a collection of iid random variables.

We have to think more for the other direction because we can't take a union bound.
\begin{align}
\Pj\pa{\max_{v\in V}d_v > (1-\ep)\fc{\ln n}{\ln \ln n}}\xra ? 0.
\end{align}•
We know how to bound for a single vertex; we think they are more or less independent, and behave like a collection of iid random variables.

Partition $G$ into two sets $S,S^c$ of size $\fc n2$. For $v\in S$, let $\wt d_v:=\#\set{u\in S^c}{(u,v)\in E}$. 
(The difference between a graph with $n$ vs. $\fc n2$ vertices is not very much.)
We have $\wt d_v \sim \text{Bin}(\fc n2, \fc dn)$ is iid. Since this is approximately Poisson,
\begin{align}
\Pj\pa{\wt d_v>(1-\ep)\fc{\ln n}{\ln \ln n}} &\sim n^{-(1-\ep)+o(1)}\\
\Pj\pa{\forall v\in S, \wt d_v\le (1-\ep) \fc{\ln n}{\ln \ln n}} &\le \pa{1-\fc{n^{\ep+o(1)}}{n}}^{\rc n2}\to 0
\end{align}
%$e^{-n^\ep}$
In summary, we created some proxies that were independent, that doesn't change the answer.

Hence $\fc{\max d_v}{\ln n/\ln\ln n}\to 1$.

We showed multiplicative fluctuations tend to 0. In fact, $\max d_v$ takes 1 of 2 values with probability taking to 1.

How about going further out? Is a vertex in a cycle of length $k$?
%any 5 vertices in any order, equally likely to be a $k$-cycle.

Let $C_k$ be the number of $k$-cycles in $G$. There are $\fc{n\fp{k}}{2k}$ ways of choosing $k$ vertices for a cycle (order matters); divide by $2k$ for symmetries. So
\begin{align}
\E C_k &= \pf{d}{n}^k \fc{n\fp k}{2k}\xra{n\to \iy} \fc{d^k}{2k}
\end{align}
which doesn't scale as $n$.
Most vertices are not in a $k$-cycle, so locally, the graph looks like trees (locally treelike). This will be important in lots of things we do for the rest of the semester. 

Maybe Erd\H os-Renyi graphs aren't the best models. It's not a good model for a social network unless you chose all your friends uniformly at random. If you have two friends, they're more likely to be friends with each other than uniformly chosen people.

There are a lack of good network models that incorporate structures we expect to see.

The law of $C_n$ is Poisson.

%What is the law of $C_n$? 
What about containment of more complicated graphs? Let $H$ be a connected graph. Then 
\begin{align}
\E C_H &\approx n^{V(H)}\pf{d}{n}^{E[H]}.
\end{align}
%constant number of time cycle
We get lots of trees, some cycles, and no more complicated graphs as $n\to \iy$.

